\documentclass[12pt]{article}
\usepackage[utf8]{inputenc}
\usepackage{cite}
\usepackage{indentfirst}

\title{Flagellent Vision Statement}
\author{weeksr and denekec}
\date{January 11, 2018}


\begin{document}

\maketitle

\begin{abstract}
Have you ever found your productivity declining due to being frequently distracted by your phone? Have you ever been browsing your Facebook feed and thought to yourself “Why am I wasting my time on this site?” Prepare to suffer even more from using these apps!


Introducing Flagellant! It’s the modern day equivalent of the 14th century practice of whipping oneself to atone for sin! Except instead of your pitiful flesh, the target will be your pitiful bank account balance.
The user sets an amount of time they would like to stay focused. During this period, every time you open one of these “time wasting apps”, Flagellant will automatically take money out of your bank account and donate it to charity. You choose the charity and the amount per click, and Flagellant does the rest!
\end{abstract}

\section{Why Flagellant\textsuperscript{TM}  is needed}
The age of smartphones is upon us, and with it have come many changes. One of the most notable is competition for people’s attention. There are thousands of apps that earn money by gaining and keeping your attention for as long as possible. So long as they have your attention, they can show you ads or encourage you to make microtransactions. \\


These companies have gotten incredibly good at gaining and keeping our attention. As Jeffrey Hammerbacher once said, “The best minds of my generation are thinking about how to make people click ads.” This techniques employed in pursuit of this goal are becoming more and more effective. And the effects of this arms race for attention has negative consequences for our mental health.\cite{forbes}
 \\


Users need tools to fight back! That is exactly what Flagellant does: it gives the user an effective way to control their own behavior and avoid mindless scrolling and swiping their time away. \\


For college students, staying focused and motivated is a constant struggle. Maybe you’ve finally achieved a blissful state of workflow but then you receive a useless notification from Facebook. Now your flow is obliterated and you find yourself wasting precious time scrolling through the dregs of your Facebook feed. \\


People are driven by punishment and reward. FlagellentTM punishes the user and rewards the charity of their choice. In this way, a simple app can solve both a user’s personal problem (their attention span) and large scale, real world problems (like world hunger) by donating to charity.and more effective. \\

\section{ Flagellant\textsuperscript{TM} achieves market differentiation with ease}
Our approach to reducing our user’s use of time-wasting apps is to decrease the incentive to use such applications by forcing them to pay money every time they do. Since one of the primary draws of these apps is that they are free to use, it seems reasonable that making users pay to use them will decrease the rate at which they are used.
There are many productivity apps out there but none punish the user in such a direct and real way. For example, there’s a productivity app where a user sets an amount of time they would like to focus for. During that time a virtual tree starts growing. If during this time slot, the user opens up any time-killing apps, the virtual tree dies. This app can be effective but the user isn’t being punished or rewarded in a real way. The threat of having a virtual tree die may not weigh heavily enough on a user’s conscious to stop them from mindlessly swiping on tinder. All of that to say, the reward and punishment from this app is only virtual. FlagellantTM is superior based on the very real impact the punishment will have on their bank account. \\


Our app will follow in the footsteps of other tools used by people hoping to control their behavior such as K9 web blocker, a browser extension that has 438,000 downloads on CNET. \\

Flagellant will not be without its limitations. The greatest limitation of our approach is that it is completely voluntary. A user will be able to select the apps that they are forced to donate money in order to use, and the user will have the option to exclude that app from the list of “time-wasting” apps. So the goal of our app is not to completely eliminate use of these apps, or force people not to use them, but rather to give our users another tool to help them control their usage of these apps. \\


The other possible limitation of our approach is that users will likely find a way to game the controls. For example, if we charge users 5 cents every time they open a time wasting app, they may simply keep the app open at all times and switch between apps using the multi-tasking buttons. Addressing these corner cases will require more programming. \\

\section{Development Roadmap}
The biggest challenge of this project is that it will require people working on it to learn how to develop an Android application. We will minimize this risk by making sure all teams working on this project have access to tutorials on this subject, and by finding someone who has developed an Android application before and ensuring they are available for consultation by students. \\


We plan to use the PayPal API to transfer money from user to charity. We will need some way to hook system calls to open other apps, which will probably be difficult, and may introduce security vulnerabilities into our user’s phones. But other than introducing potentially catastrophic security flaws, our idea is flawless! \\

\bibliography{myref}
\bibliographystyle{plain}

\end{document}
